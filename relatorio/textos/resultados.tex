\section{Resultados} \label{sec:resultados}

\subsection{Rotações}

    \begin{figure}[H]
    \centering
    \begin{subfigure}{0.25\textwidth}
        \centering
        \includegraphics[width=0.9\textwidth]{rotacoes/16_alp_viz.png}
        \caption{~\texttt{vizinho}.}
    \end{subfigure}%
    \hspace{8pt}%
    \begin{subfigure}{0.25\textwidth}
        \centering
        \includegraphics[width=0.9\textwidth]{rotacoes/16_alp_bil.png}
        \caption{~\texttt{bilinear}.}
    \end{subfigure}
    \\[8pt]
    \begin{subfigure}{0.25\textwidth}
        \centering
        \includegraphics[width=0.9\textwidth]{rotacoes/16_alp_bic.png}
        \caption{~\texttt{bicubica}.}
    \end{subfigure}%
    \hspace{8pt}%
    \begin{subfigure}{0.25\textwidth}
        \centering
        \includegraphics[width=0.9\textwidth]{rotacoes/16_alp_lag.png}
        \caption{~\texttt{lagrange}.}
    \end{subfigure}

    \caption{Rotação de 15\textdegree{} no plano da imagem aplicada em \texttt{house16.png} ($16 \times 16$).}
    \label{fig:house16:alp}
\end{figure}

    Na \cref{fig:rot:house64}, podemos ver com clareza a diferença entre os métodos. Nas interpolações bilinear e bicúbica, a imagem resultante aparece com um pequeno borramento. Esse efeito é bem mais fraco com polinômios de Lagrange.

    Para o aproximação por vizinho mais próximo, a figura fica com um serrilhado bem presente, principalmente nas bordas. Isso aparece inclusive em imagens grandes, como na \cref{fig:rot:house}. Além desse método, quase não existem diferenças visuais para as imagens de $512 \times 512$.

    \begin{figure}[H]
    \centering
    \begin{subfigure}{0.3\textwidth}
        \centering
        \includegraphics[width=0.8\textwidth]{rotacoes/64_alp_viz.png}
        \caption{~\texttt{vizinho}.}
    \end{subfigure}%
    \hspace{8pt}%
    \begin{subfigure}{0.3\textwidth}
        \centering
        \includegraphics[width=0.8\textwidth]{rotacoes/64_alp_bil.png}
        \caption{~\texttt{bilinear}.}
    \end{subfigure}
    \\[8pt]
    \begin{subfigure}{0.3\textwidth}
        \centering
        \includegraphics[width=0.8\textwidth]{rotacoes/64_alp_bic.png}
        \caption{~\texttt{bicubica}.}
    \end{subfigure}%
    \hspace{8pt}%
    \begin{subfigure}{0.3\textwidth}
        \centering
        \includegraphics[width=0.8\textwidth]{rotacoes/64_alp_lag.png}
        \caption{~\texttt{lagrange}.}
    \end{subfigure}

    \caption{Rotação de -30\textdegree{} em torno do eixo Y aplicada em \texttt{house64.png} ($64 \times 64$).}
    \label{fig:rot:house64}
\end{figure}

    \begin{figure}[H]
    \centering\hfill
    \begin{subfigure}{0.4\textwidth}
        \centering
        \includegraphics[width=0.9\textwidth]{rotacoes/house_alp_viz.png}
        \caption{~\texttt{vizinho}.}
    \end{subfigure}%
    \hfill%
    \begin{subfigure}{0.4\textwidth}
        \centering
        \includegraphics[width=0.9\textwidth]{rotacoes/house_alp_bil.png}
        \caption{~\texttt{bilinear}.}
    \end{subfigure}\hfill
    \\[8pt]\hfill
    \begin{subfigure}{0.4\textwidth}
        \centering
        \includegraphics[width=0.9\textwidth]{rotacoes/house_alp_bic.png}
        \caption{~\texttt{bicubica}.}
    \end{subfigure}%
    \hfill%
    \begin{subfigure}{0.4\textwidth}
        \centering
        \includegraphics[width=0.9\textwidth]{rotacoes/house_alp_lag.png}
        \caption{~\texttt{lagrange}.}
    \end{subfigure}\hfill

    \caption{Rotação de 15\textdegree{} no plano da imagem em \texttt{house.png}.}
\end{figure}

\subsection{Escalonamento}

    \begin{figure}[H]
    \centering
    \begin{subfigure}{0.3\textwidth}
        \centering
        \includegraphics[width=0.9\textwidth]{escala/city_13_viz.png}
        \caption{~\texttt{vizinho}.}
    \end{subfigure}%
    \hspace{8pt}
    \begin{subfigure}{0.3\textwidth}
        \centering
        \includegraphics[width=0.9\textwidth]{escala/city_13_bil.png}
        \caption{~\texttt{bilinear}.}
        \label{fig:esc:13:bil}
    \end{subfigure}
    \\[8pt]
    \begin{subfigure}{0.3\textwidth}
        \centering
        \includegraphics[width=0.9\textwidth]{escala/city_13_bic.png}
        \caption{~\texttt{bicubica}.}
    \end{subfigure}%
    \hspace{8pt}%
    \begin{subfigure}{0.3\textwidth}
        \centering
        \includegraphics[width=0.9\textwidth]{escala/city_13_lag.png}
        \caption{~\texttt{lagrange}.}
    \end{subfigure}

    \caption{Escalonamento com $S_x = S_y = 1/3$ aplicado em \texttt{city.png} ($512 \times 512$).}
    \label{fig:esc:13}
\end{figure}

    \begin{figure}[H]
    \centering
    \begin{subfigure}{0.3\textwidth}
        \centering
        \includegraphics[width=0.9\textwidth]{escala/128_15_viz.png}
        \caption{~\texttt{vizinho}.}
    \end{subfigure}%
    \hspace{8pt}
    \begin{subfigure}{0.3\textwidth}
        \centering
        \includegraphics[width=0.9\textwidth]{escala/128_15_bil.png}
        \caption{~\texttt{bilinear}.}
        \label{fig:esc:15:bil}
    \end{subfigure}
    \\[8pt]
    \begin{subfigure}{0.3\textwidth}
        \centering
        \includegraphics[width=0.9\textwidth]{escala/128_15_bic.png}
        \caption{~\texttt{bicubica}.}
    \end{subfigure}%
    \hspace{8pt}%
    \begin{subfigure}{0.3\textwidth}
        \centering
        \includegraphics[width=0.9\textwidth]{escala/128_15_lag.png}
        \caption{~\texttt{lagrange}.}
    \end{subfigure}

    \caption{Escalonamento com $S_x = S_y = 1.5$ aplicado em \texttt{city128.png} ($128 \times 128$).}
    \label{fig:esc:15}
\end{figure}

    \begin{figure}[H]
    \centering
    \begin{subfigure}{0.33\textwidth}
        \centering
        \includegraphics[width=0.8\textwidth]{escala/128_86_viz.png}
        \caption{~\texttt{vizinho}.}
    \end{subfigure}%
    \hspace{8pt}
    \begin{subfigure}{0.33\textwidth}
        \centering
        \includegraphics[width=0.8\textwidth]{escala/128_86_bil.png}
        \caption{~\texttt{bilinear}.}
    \end{subfigure}
    \\[8pt]
    \begin{subfigure}{0.33\textwidth}
        \centering
        \includegraphics[width=0.8\textwidth]{escala/128_86_bic.png}
        \caption{~\texttt{bicubica}.}
    \end{subfigure}%
    \hspace{8pt}%
    \begin{subfigure}{0.33\textwidth}
        \centering
        \includegraphics[width=0.8\textwidth]{escala/128_86_lag.png}
        \caption{~\texttt{lagrange}.}
    \end{subfigure}

    \caption{Redimensionamento para $80 \times 60$ aplicado em \texttt{city128.png} ($128 \times 128$).}
    \label{fig:esc:86}
\end{figure}

\subsection{Reconstrução}

\subsection{Tempo de Execução}
